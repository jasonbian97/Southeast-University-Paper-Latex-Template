\chapter{总结与展望}
\section{总结}
白斑面积是临床治疗白癜风效果的重要评价指标, 由于在评估方法上缺乏共识,这使得分析或比较不同研究的结果变得困难。而测量白斑面积的基础则是准确将白斑区域分割出来。

本文首先对现有的白癜风评价体方法展开充分的研究,进行了综合的对比,指出主观或半客观评价体系的缺点,并阐明了客观评价体系的必要性。就白斑分割方法而言,首先指出了分割在主观评价体系中的重要性,并从两个角度进行了相关工作的介绍,即色素性皮肤病分割与白癜风图像的分析与分割,并针对目前的所使用的方法进行了细致的原理分析和优缺点对比。分析得出了以白癜风为代表的色素性皮肤病所具有的特点及难点。

为了应对这些缺陷和难点,本文首先提出了到目前为止最大的一个白癜风数据集(Vit2019),此数据集拥有1000张来自临床和网络的白癜风图片,并且均进行了白癜风区域的像素级标注。而且本文从收集、标注、多样性与挑战四个方面详细介绍了该数据集,并举例分析了部分样例图片。这一数据集为之后的强监督分割与弱监督分割部分的效果评价提供了真值。

在对白癜风图像预处理的过程中,结合其图像对比度低、皮损与正常皮肤过渡区域模糊等特点,提出了使用超像素分割方法作为图像预处理步骤,从而达到降低维度、剔除异常像素点、保留较完整准确的皮损边界等三个目的;由于图像大小尺寸不一致,引出了经典超像素分割算法中初始种子点数目确定难的问题,并针对其提出了改进的方法。

对于强监督意义下的分割框架,本文着重介绍了一种经典的、具有代表性的网络结构——Unet,并且从定量和定性的角度展现并分 析了 Unet 在 Vit2019 上的效果。

对于弱监督分割框架,本章首先分析了在医疗图像处理领域对图像真值标注的困难性以及强监督分割所面临的限制。然后提出了本文的方法,并详细介绍了每一步骤中的原理以及意义,将 “既见森林,又见树木”的策略引入分割过程,使得该方法既可以从语义层面获得白癜风自身的特征,又可以从微观层面针对每一个患者的皮肤状况作出推理。而且在该方法中,我们结合了反馈的思想来控制显著性传播,使其可以根据不同情况自适应的调整传播过程。

最后通过多个实验验证了该弱监督分割框架的有效性,并与强监督学习的实验结果进行对比,发现在一些拍照环境恶劣,对比度很低的情况下,本文提出的方法甚至可以更好的分割白癜风区域并保持白癜风的边缘细节。

\section{展望}
本文提出的基于显著性传播的弱监督分割方法,在取得优良的性能的同时,也存在着局限性。由于临床拍摄环境的复杂,在某些极端情况下分割效果仍然难以尽如人意。例如在高光条件下或者当皮损区域位于人脸鼻侧附近,其崎岖的边界会对显著性传播造成一定的影响。另一方面,本文所提出的方法同样具有拓展到其他皮损区域分割的潜力,例如对黄褐斑、烧伤表皮进行分割。

未来在进行白癜风患病程度评价时,对于面积的计算可以采取两种策略,一是计算相对面积,二是计算绝对面积,此处以人脸处的白癜风面积为例,对未来的工作进行展望。计算相对面积,即通过分割方法得到掩模图像后,计算其中白色区域(白癜风)占整个图像区域的百分比;为了计算绝对面积,由于拍摄的角度不同,在二维图像中的白癜风区域受到了不同程度的形变,因此直接通过二维图像计算物理面积较不准确。因此,可以先通过3D建模技术构建人脸模型,然后将分割好的掩模图像投到人脸模型上,最终通过计算三角面片数来获得白癜风的物理面积。

通过对白癜风真实物理面积或者比例的计算,使得此模型更加完善,便于临床使用和推广。







